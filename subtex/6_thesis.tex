%% \section{T\'ezispontok}

\section{Playground}

\begin{enumerate}
\item Thesis: Thesis: Constrained, explicit predictive control for current source buck-type rectifiers.\\
    The proposed CSR model has been examined from the design and implementation points of view with the purpose of explicit model-based predictive control, and It proved to be that regular set of differential equations of the CSR appears to be too complex for such a design approach. To address this issue the usage of separated AC and DC equation sets was implemented. This solution eliminates bilinearity and enables the application of linear control design techniques, and enables different solutions on AC and DC side. For the modulation, Current-based SVPWM has been used with an emphasis on the reduction of switching losses. For DC side control explicit model predictive control method is described and the method's effectiveness was compared to conventional state feedback control. The implementation was carried out in a Matlab/Simulink environment and the proposed control structure has been tested by simulation. Additionally, the proper implementation of the system in a modern DSP chip would result in real-time operation.
\item Thesis: Geometrical indicator for voltage unbalance in three phase networks.\\
    The currently used measures of voltage unbalance has been extended with a norm candidate. It is more demanding from the computational point of view but has a new feature namely it checks electrical asymmetry, i.e. the norm of a $\pm120$ degree rotated version of the ideal three phase phasor is zero in the geometrical sense. The defined norm is applied as a cost function in the asymmetry reducing controller structure also presented in the paper. Simulations show that the geometrical based unbalance indicator can serve as a basis of further research.
\item Thesis: Voltage unbalance compensation with optimization based control algorithm and asymmetrical inverter structure.\\
    The suggested controller structure enables the residential users owning a grid synchronized domestic power plant to reduce voltage unbalance measurable at the connection point. The fundamental element of the system is a modified three phase inverter that is capable of the asymmetric injection of any current waveforms to the network. The optimization based control algorithm injects the available energy (as current waveform) in such a way, that the voltage unbalance decreases. This optimization problem is usually constrained by the available renewable energy supplied by the power plant.\\
    The control structure has been tested on a low voltage network model in a dynamical simulation environment consisting of the models of the electrical grid, a domestic power plant, asymmetrical inverter circuit, and different types of loads. Different simulation experiments has been run for each norm and for both the power constrained and unconstrained case. The preliminary results show that this structure can serve as a residential level voltage quality improvement method for the three phase low voltage network.
\end{enumerate}
