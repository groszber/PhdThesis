%% \section{T\'ezispontok}

\section{Playground}

\begin{enumerate}
\item Thesis: Geometrical indicator for voltage unbalance in three phase networks.\\
    I extended the currently used measures of voltage unbalance with a new norm candidate. I found out that it is more demanding from the computational point of view, but has a new feature namely it checks electrical asymmetry, i.e. the norm of a $\pm120$ degree rotated version of the ideal three phase phasor is zero in the geometrical sense. I compared my geometrical approach to the standard wide-spread use of voltage unbalance indicator (TDV) and found out it carries additional information, whilst retaining it's original purpose. 
				
\item Thesis: Voltage unbalance compensation with optimization based control algorithm and asymmetrical inverter structure.\\
     I found out that the regular current controlling applications would not fit to the purpose for reducing voltage unbalance whilst only relying on the voltage measurement. As such I developed an asymmetrical current source inverter (ACSI) circuit with combined asynchronous parallel pattern search (APPS) control structure, in Matlab/Simulink environment, and I applied the my geometrical norm as a cost function. I showed with validating simulations, that the geometrical based unbalance indicator can serve as a basis of further research. \\
		The fundamental element of the system is a modified three phase inverter that is capable of the asymmetric injection of any current waveforms to the network. The optimization based control algorithm injects the available energy (as current waveform) in such a way, that the voltage unbalance decreases. This optimization problem is usually constrained by the available renewable energy supplied by the power plant. This suggested controller with combined ACSI structure enables the residential users owning a grid synchronized domestic power plant to reduce voltage unbalance measurable at any low voltage domestic the connection point.\\
    I also tested the control structure on a real low voltage network model in a dynamical simulation environment consisting of the models of the electrical grid, a domestic power plant, ACSI, and different types of loads. Different simulation experiments has been run for each norm and for both the power constrained and unconstrained case. I showed with the evaluation that this structure can serve as a residential level voltage quality improvement method for the three phase low voltage network.
		
\item Thesis: Constrained, explicit predictive control for current source buck-type rectifiers.\\
    I proposed a simplified CSR model, which was derived from the well known CSR structure and was examined from design and implementation points of view with the purpose of explicit model-based predictive control. The regular set of differential equations of the CSR appeared to be too complex for my a design approach, for applying explicit predictive control. I adressed this issue with the separation AC and DC equation sets was of the CSR to decrease complexity and easy controller design. With this solution I eliminated bi-linearity and enabled the application of linear control design techniques. I used current-based SVPWM the modulation, what has been used with an emphasis on the reduction of switching losses. For DC side control I implemented explicit model predictive control (EMPC) and I compared this method's effectiveness to conventional state feedback control (SFC). I implemented the CSR structure and the proposed controller with EMPC on DC and active damping on the AC side in Matlab/Simulink environment and tested by simulation. Additionally, I tested the proper implementation's computational requirements in a modern DSP chip, which would serve in real-time operation.
\end{enumerate}
