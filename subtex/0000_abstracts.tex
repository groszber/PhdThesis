\thispagestyle{plain}
\chapter*{Compendiary}
 (max 2500 karakter)

\newpage

\thispagestyle{plain}
\chapter*{Abstarct}
 (max 8-10 sor)
This Thesis is consisted about optimal control of power converters. The firs part is consisting of a constrained optimal control of a current source rectifier (CSR) is presented, based on a mathematical model developed in Park's frame. To comply with the system constraints an explicit model-based predictive controller was established. To simplify the control design, a disjointed model was utilised due to the significant time constant differences between the AC and DC side dynamics. As a result, active damping was used on the AC side, and explicit Model Predictive Control (MPC) on the DC side. The results are compared by simulation with the performance of a state feedback control. This is followed by developing a cost function for mitigating voltage asymmetry on a domestic network, which requires only measuring the voltage whist only current interventions are required. Lastly an asymmetric inverter structure was developed to serve the const function minimizing compensational needs. Results were implemented and validated with Matlab simulations.

%\newpage
%
%\thispagestyle{plain}
%\chapter*{ (3. nyelven) \normalfont{抽象}}
%(max 8-10 sor)


