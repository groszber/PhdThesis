\chapter*{Abstract}
\thispagestyle{plain}

Starting the third decade after the millennium, in the European Union, the approach of the energy-, and with it the automotive industries has changed significantly since. The increasing governmental interventions on the industry, started with the bad reputation of diesel in passenger cars, and increasingly tighter restrictions on lead battery technologies, also Germany's renewable policies, dismantling nuclear plants encourage manufacturers and researchers for innovation in the direction of power efficiency. Solutions, which seemed to occupy the second place on the podium, have been re-assessed and re-thought in the light of this approach in terms of power topologies and algorithms alike.\\
As the tide of renewable sources has taken the western Europe with a storm, with the high anticipation of getting rid of fossil fuels, it brought down a myriad of problems to solve. The first of these is the high stochastic nature of these source, with the ever increasing demand on storage, to smoothen out the arms of the scale of supply and demand. The second comes with the incentive (policy maker, and civic alike in pursuit of "cheap" and non polluting energy) of installing these sources in residual areas, parking lots, schools, household roofs, just to name a few. On top of that, these actors have higher consumers in they disposal, using them on they whim, making the situation even worse.\\
This introduces two interesting, yet completely understandable phenomena, which act as the "symptoms" on the quality of our well accessible electric power for our consumers, due to the reason of these trends. The first is the harmonic distortion of the network's voltage phases, tackled with load regulations, and a wide range of active and passive filter designs. The second is the voltage phase asymmetry alias the voltage unbalance, appearing when uneven production, consumption, or distribution on the network.\\
This is a major yet often overlooked power quality problem in low voltage residential feeders due to the random location and rating of single-phase renewable sources and uneven distribution of household loads. Here a new indicator of voltage deviation is proposed that may serve as a basis of analysis and compensation methods in this dimension of power quality. The first half of this dissertation is about indicating such voltage asymmetry and mitigating it with a complex controller and electric power conversion structure, which is integrated with an optimization based control algorithm that uses asynchronous parallel pattern search as its engine. This structure uses current control on each phase to achieve the results, on the voltage quality. Additionally, the calculations showed, that the implementation could achieve noteworthy reductions in $CO_2$ emissions for an average household. \\
The second half is zooming down on the current control aspect itself, where a model of a power converter device is used for an optimal, predictive controller structure, to achieve the demanded efficiency, also with the calculation capacity in check. Starting with the design of a constrained optimal control of a current source rectifier, based on a mathematical model developed in Clarke and Park frame. Despite the clean establishment of the model's differential equations, the underlying bilinearity would make further controller design complicated. To comply with the system constraints an explicit model-based predictive controller was established. To simplify the control design, and eliminate the bilinearity from the equation system, a disjointed model was utilised due to the significant time constant differences between the AC and DC side dynamics. As a result, active damping was used on the AC side, and explicit model based predictive control on the DC side. The results are compared by simulation with the performance of a state feedback control.\\

  %The paper proposes three main results. First of all a novel voltage norm capable of indicating unbalance and \textcolor{red}{under-voltage in a single value. Afterwards,} a three phase unbalance reduction controller structure is given. As the third main result,  The suggested structure and the underlying three phase power grid model has been implemented in a dynamical simulation environment and tested against engineering expectations.





%    The simulation based experiments served as a proof of concept for the proposed complex control structure. The experiments included performance and robustness analysis, both of them concluded that the proposed control and inverter structure is promising.
%    The proposed three phase inverter structure together with the control algorithm connected with a renewable source (photovoltaic panel or wind turbine) is capable of an asymmetric power injection \textcolor{magenta}{or rerouting the energy flow} to the grid so that the voltage unbalance decrease. This is also important from the environmental point of view since the achieved power loss reduction can easily be translated to $\textnormal{CO}_2$ emission reduction and carbon footprint - these indicators has also been calculated.===\\


   % This Thesis is consisted about optimal control of power converters.  This is followed by developing a cost function for mitigating voltage asymmetry on a domestic network, which requires only measuring the voltage whist only current interventions are required. Lastly an asymmetric inverter structure was developed to serve the const function minimizing compensational needs. Results were implemented and validated with Matlab simulations.

%\newpage
%
%\thispagestyle{plain}
%\chapter*{ (3. nyelven) \normalfont{抽象}}
%(max 8-10 sor)


\chapter*{Tartalmi Kivonat}
\thispagestyle{plain}

\chapter*{Abstrakt}
\thispagestyle{plain}



