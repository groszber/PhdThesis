\chapter{Thesis and Summary}

\section{Summary}

Write summary here..

\section{Developed thesis}

\begin{enumerate}[a.)]
\item\textbf{Thesis}: \emph{Constrained, explicit predictive control for current source buck-type rectifiers}\\
    I proposed a simplified CSR model, which was derived from the well known CSR structure and was examined from design and implementation points of view with the purpose of explicit model-based predictive control. The regular set of differential equations of the CSR appeared to be too complex for my a design approach, for applying explicit predictive control. \\
		I adressed this issue with the separation AC and DC equation sets was of the CSR to decrease complexity and easy controller design. With this solution I eliminated bi-linearity and enabled the application of linear control design techniques. I used current-based SVPWM the modulation, what has been used with an emphasis on the reduction of switching losses. \\
		For DC side control I implemented explicit model predictive control (EMPC) and I compared this method's effectiveness to conventional state feedback control. I implemented the CSR structure and the proposed controller with EMPC on DC and active damping on the AC side in Matlab/Simulink environment and tested by simulation. Additionally, I tested the proper implementation's computational requirements in a modern DSP chip, which would serve in real-time operation.\\
		
\item\textbf{Thesis}: \emph{Geometrical indicator for voltage unbalance in three phase networks}\\
    I extended the currently used measures of voltage unbalance with a new norm candidate. I found out that it is more demanding from the computational point of view, but has a new feature namely it checks electrical asymmetry, i.e. the norm of a $\pm120$ degree rotated version of the ideal three phase phasor is zero in the geometrical sense. I compared my geometrical approach to the standard wide-spread use of voltage unbalance factor (VUF) and found out it carries additional information, whilst retaining it's original purpose.\\
		
\item\textbf{Thesis}: \emph{Voltage unbalance compensation with optimization based control algorithm and asymmetrical inverter structure}\\
     I found out that the regular current controlling applications would not fit to the purpose for reducing voltage unbalance whilst only relying on the voltage measurement. As such I developed an asymmetrical current source inverter (ACSI) circuit with combined asynchronous parallel pattern search (APPS) control structure, in Matlab/Simulink environment, and I applied the my geometrical norm as a cost function. I showed with validating simulations, that the geometrical based unbalance indicator can serve as a basis of further research. \\
		The fundamental element of the system is a modified three phase inverter that is capable of the asymmetric injection of any current waveforms to the network. The optimization based control algorithm injects the available energy (as current waveform) in such a way, that the voltage unbalance decreases. This optimization problem is usually constrained by the available renewable energy supplied by the power plant. This suggested controller with combined ACSI structure enables the residential users owning a grid synchronized domestic power plant to reduce voltage unbalance measurable at any low voltage domestic the connection point.\\
    I also tested the control structure on a real low voltage network model in a dynamical simulation environment consisting of the models of the electrical grid, a domestic power plant, ACSI, and different types of loads. Different simulation experiments has been run for each norm and for both the power constrained and unconstrained case. I showed with the evaluation that this structure can serve as a residential level voltage quality improvement method for the three phase low voltage network.\\
\end{enumerate}
		\section{Applications and future work}
		
		In this section the possible applications shall be described based on my thesis. These are not yet scope of current research activities, but can serve as a potential direction and evolution for these results.
		
		\subsection{Geometrical voltage unbalance norm}
		
		As mentioned the geometrical norm's largest weakness is the computational demand. The required areas, computed from the voltage phasors realized with the corresponding Matlab functions, which are not designed for continuous calucation, especially not for time constants for power electronic devices. This can be resolved, via replacing the calculation of the symmetrical difference (difference between the two polygon's union and intersection) with finite-elmenet method, with scalability, where the segmentation's resolution would be adjustable and based on the corresponding simulation's (or system's) time constant and the simulator machine's calculation capabilities. After this, the calculation method shall be phrased in an traditional equation form, using the toolset of set theory, and linear algebra. This way, the norm's calculation could be further reduced, and could be implemented on a cheaper embedded system.\\
		The geometrical norm's usefulness was already proven compared o the regular $VUF$ method. However the robustness of the method seems implicitly proven, is shall be tested on real conditions, even in extreme (faulty) situations, and with such hard constrains could be defined, where the algorithm outlives its usefulness. This way, instead of just a resulting number, a full formulation of an optimization problem could be utilized, presented in section \ref{BASICCSR:sec:OptimalControl}, with model based predictive capabilities. This can be further enhanced, with recognizing different scenarios, form general inefficiency, to fault prediction, or handling (like graceful degradation).
		
		\subsection{Voltage unbalance reducing inverter structure}
		
		The inverter structure employ a variety of subsystems, which shall work together in harmony. A global optimum shall be defined (with weighting the various factors) based on external circumstances, and the customers needs. This way, the individual Kirchhoff equations based differential equations shall be established, and controller designed.\\
		This way the system's physical properties shall be scalable (based on a household's needs and its energy producing and storing capabilities), as in designing a real power electric system for implementation. Next actual implementation shall be proceeded, with prototype realization on a test bench, with a simulated domestic network connection. \\
		The APPS method, however fulfills its optimizing responsibilities very well in such an environment, where changes are expected to be highly stochastic, based on network knowledge, a netwok- and/or device model based optimal control shall be established, where various current and voltage inertias can be taken into acount, giving a leverage for prediction. 
		
		\subsection{Explicit model predictive control for buck-type rectifier}
		
		From mathematical perspective it is an alternative possibility to take the harder route and take the inherent bilinearity into account. This way, the devices equations should not be partitioned, and a hybrid design can be commenced. This has been performed in the literature, but seldom on three phase systems, for complexity reasons.\\ The topology could be altered, by removing (or greatly decreasing) some of the device's filtering  capabilities, in inductive or capacitive terms (some inductors, like the choke $L_{dc}$ are non-removable), and try to outsource this problems to the controller itself to some degree.\\
		Try the cost function from Euclidean norm to an infinity norm, and also give stricter constrains. The optimum could be power throughput based instead of a specified current. The types of loads can be extended, and merged with the current equation system. This enables to expand the research on electric machinery, where starting/ breaking dynamics can be tested. On the other side, the effects on the supplying network can be taken into consideration, further reducing harmonics, and test conditions in the presence of unbalance.
		Lastly, experimental results shall be performed on a device, further validating its usefulness, and implement the used S-function from Simulink to an embedded system.
